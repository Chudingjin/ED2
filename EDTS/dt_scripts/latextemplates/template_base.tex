\documentclass[12pt]{beamer}
\useoutertheme{infolines}
\usetheme{Boadilla}
\usepackage{beamerthemesplit}
\usepackage{amsmath}
\usepackage{graphicx}
\usepackage{framed,color}
\usepackage{minibox}
\usepackage{picture,xcolor}
\setbeamertemplate{caption}[numbered]
\setbeamerfont{caption}{size=\scriptsize}
%\setbeamertemplate{navigation symbols}{}
\usecolortheme{beaver}

%\usebackgroundtemplate
%

\title[ED 2.2, Scale \& Biophysics]{Shop Talk: Biophysics and Patch Dynamics in the Ecosystem Demography Model 2.2}

\author{Ryan Knox}
\institute[LBNL]{
  LBNL \\
  CESM Annual Workshop - Land Modeling Working Group \\
  Brekenridge CO}
\date{June 19, 2013}

\begin{document}
{
  \usebackgroundtemplate{\includegraphics[width=\paperwidth,height=\paperheight]{Images/P1000750_nocont.eps}}
  \frame{
    \titlepage
  }
}

\section{Overview}

\frame{
\begin{center}
ED?
\end{center}
\setlength{\unitlength}{1cm}
\begin{figure}[hp]
\begin{center}
  \begin{picture}(11,7)(0,0)
    \setlength\fboxsep{0pt}
    \scriptsize
    \put( 0.25,0.25){\colorbox{green!10}{\framebox(5,6)
        {\minibox{ Terrestrial Ecosystem Dynamics \\
                   \quad Size-Age-Structure \\
                   \quad \quad Approximation\\
                   \quad Photosynthesis \\
                   \quad Growth \& Allocation \\
                   \quad Allometry \\
                   \quad Mortality \\
                   \quad Disturbance \\
                   \quad Recruitment }}}}
    \put( 5.75,0.25){\colorbox{blue!10}{\framebox(5,6)
        {\minibox{ LSM/SVAT/Biophysics Stuff
                   }}}}
  \end{picture}
\end{center}
\end{figure}
}  

\frame{
  \begin{figure}[!ht]
    \begin{center}
      \includegraphics[height=0.95\textheight]{Images/mlo_submodels.eps}
    \end{center}
  \end{figure}
}

%\frame{
%\begin{block}{Development History}
%  \scriptsize
%Concepts Ecosystem Scaling: [Hurtt et al, 1998] \\
%Dynamic Vegetation Model, ED1: [Moorcroft et al, 2001]\\
%Land Model, ED 2: [Medvigy et al, 2009] \\
%\end{block}
%
%\setlength{\unitlength}{1cm}
%\begin{figure}[hp]
%\begin{center}
%  \begin{picture}(10,7)(0,0)
%    \setlength\fboxsep{0pt}
%    \scriptsize
%    \put( 0.00,1.5){\colorbox{green!20}{\framebox(3,1.25){\minibox{Parameter\\Optimization}}}}
%    \put( 3.33,1.5){\colorbox{green!20}{\framebox(3,1.25){Allometry}}}
%    \put( 6.66,1.5){\colorbox{green!20}{\framebox(3,1.25){\minibox{OLAM\\BRAMS}}}}
%    \put( 0.00,3.0){\colorbox{green!20}{\framebox(3,1.25){Vectorization}}}
%    \put( 3.33,3.0){\colorbox{green!20}{\framebox(3,1.25){Disturbance}}}
%    \put( 0.00,4.75){\colorbox{green!20}{\framebox(3,1.25){PFTs}}}
%    \put( 3.33,4.75){\colorbox{green!20}{\framebox(3,1.25){\minibox{Massively\\Parallel}}}}
%    \put( 6.66,4.75){\colorbox{green!20}{\framebox(3,1.25){HDF5}}}
%    \put( 6.66,3.00){\colorbox{green!20}{\framebox(3,1.25){Biophysics}}}
%  \end{picture}
%\end{center}
%\end{figure}
%}  

\frame{
\begin{columns}[c]
\column{0.8\textwidth}
\includegraphics[width=0.9\textwidth]{Images/scale_poly_far.eps}
\column{0.2\textwidth}
\begin{center}
  ``\emph{Polygon}''\\~\\
  Climate
  \end{center}
\end{columns}
}

\frame{
\begin{columns}[c]
\column{0.7\textwidth}
\includegraphics[width=0.9\textwidth]{Images/scale_tree.eps}
\column{0.3\textwidth}
\scriptsize
\begin{table}[h!b!p!]
  \begin{tabular}{ c }
    Meteorology \\
    \\
    \\
    \\
    Soil Properties -\\
    Terrain \\
    \\
    \\
    Disturbance - \\
    Canopy Biophysics \\
    Conservation of Mass and Energy \\
    \\
    \\
    \\
    Plant Biophysics - \\
    Mortality -\\
    Recruitment - \\
    Carbon Assimilation -\\
    Phenology\\
    \end{tabular}
  \end{table}
\end{columns}
}


%%\frame{
%%\begin{columns}[c]
%%\column{0.7\textwidth}
%%\includegraphics[width=0.80\textwidth]{Images/soil_types_circle.eps}
%%\column{0.3\textwidth}
%%  ``\emph{Site}''\\
%%  Soil Texture \\
%%  Depth to Bedrock \\
%%  Slope \\
%%  Aspect \\
%%  Basin Delineation
%%\end{columns}
%%}

%%\frame{
%%\begin{columns}[c]
%%\column{0.6\textwidth}
%%\includegraphics[width=1.0\textwidth]{Images/gap_google.eps}
%%\column{0.4\textwidth}
%\begin{center}
%%  ``\emph{Patch}''\\
%%  Disturbance \\
%%  Conservation of Mass and Energy 
%  \end{center}
%%\end{columns}
%%}

%\frame{
%\begin{columns}[c]
%\column{0.5\textwidth}
%\includegraphics[width=1.0\textwidth]{Images/dist_pie_1.eps}
%\column{0.5\textwidth}
%``\emph{Patch}''
%\scriptsize
%  \[
%    \underbrace{ \frac{\partial}{\partial t}p(a,t)}_\text{$\Delta$ age structure} =
%    \text{    }
%    \underbrace{-\frac{\partial}{\partial a}p(a,t)}_\text{aging} 
%    \text{    }
%    \underbrace{-\lambda(a,t)p(a,t)}_\text{disturbance}
%  \]
%  \\
%  \[ \]
%  \[ \int_0^{\inf} p(a,t) da = 1 \]
%\end{columns}
%}

%\frame{
%\begin{columns}[c]
%\column{0.5\textwidth}
%\includegraphics[width=1.0\textwidth]{Images/dist_pie_2.eps}
%\column{0.5\textwidth}
%  ``\emph{Patch}''\\
%\scriptsize
%  \[
%    \underbrace{ \frac{\partial}{\partial t}p(a,t)}_\text{$\Delta$ age structure} = 
%    \text{    }
%    \underbrace{-\frac{\partial}{\partial a}p(a,t)}_\text{aging} 
%    \text{    }
%    \underbrace{-\lambda(a,t)p(a,t)}_\text{disturbance}
%  \]
%  \\
%  \[ \]
%  \[ \int_0^{\inf} p(a,t) da = 1 \]
%\end{columns}
%}

%\frame{
%\begin{columns}[c]
%\column{0.5\textwidth}
%\includegraphics[width=1.0\textwidth]{Images/dist_pie_2.eps}

%\column{0.5\textwidth}
%\includegraphics[width=1.0\textwidth]{Images/dist_pie_3.eps}
%
%\end{columns}
%}



%\frame{
%\frametitle{Disturbance}
%\begin{center}
%\includegraphics[width=0.75\textwidth]{Images/disturbance_questions.eps}
%\scriptsize
%\setlength{\unitlength}{1cm}
%\begin{figure}[hp]
%\begin{center}
%  \begin{picture}(11,3)(0,0)
%    \setlength\fboxsep{0pt}
%    \put( 3.00,1.25){\colorbox{blue!20}{\framebox(5,1.52)
%        {\minibox{Distribution of Disturbance?\\Environment of post-disturbance?\\Functional Relationships?}}}}
%  \end{picture}
%\end{center}
%\end{figure}
%\end{center}
%}

%\frame{
%\frametitle{Disturbance}
%\begin{center}
%\includegraphics[width=0.9\textwidth]{Images/chambers_p3_pnas.eps}
%\end{center}
%~\\
%\scriptsize
%Chambers et al. PNAS, 110(10), 2013.
%
%}


%\frame{
%\begin{columns}[c]
%\column{0.7\textwidth}
%\includegraphics[width=0.95\textwidth]{Images/patch1_n-lai.eps}
%\column{0.3\textwidth}
%\begin{center}
%  ``\emph{Cohort}''\\
%  Plant Groups \\
%  Number Density \\
%  Size and Allometry \\
%  Functional Type (PFT) \\
%  Carbon Balance \\
%  Phenologic Status \\
%  Thermodynamics
%  \end{center}
%\end{columns}
%}

%\frame{
%  \begin{figure}[!ht]
%    \begin{center}
%      \includegraphics[height=1.00\textheight,orientation=\landscape]{Images/marcos_slide_cohorts_rotate.eps}
%    \end{center}
%  \end{figure}
%}

\section{Biophysics \& Information Structure}

\subsection{1D Canopy}

\frame{
  \begin{center}
    \includegraphics[height=0.98\textheight]{Images/patch4_n-lai_v2.eps}
  \end{center}
}

\frame{
  \begin{center}
    \includegraphics[height=0.90\textheight]{Images/patch4_intarea.eps}
  \end{center}
}

\frame{

  \frametitle{Mixed Model}
  \begin{columns}[cc]
    \column{0.57\textwidth}
    \begin{center}
      \includegraphics[width=0.99\textwidth]{Images/patch4_mixed_v2.eps}
    \end{center}
    \column{0.43\textwidth}
    \scriptsize
    Wind-Speed Scaling \\
    Turbulent Transport \\~\\
    \[
      u_{(z)}/u_{(z_c)} = e^{-n(1-\zeta_{(z)}/\zeta_{(z_c)})}
    \]
    \\~\\
    cumulative drag $\zeta$ \\
    \[ \zeta_{(z)} = \int_o^z[C_{d(z^\prime)}a_{(z^\prime)}/P_{m(z^\prime)}]dz^\prime \]
    fluid drag $C_d$ \\
    frontal area of the drag elements $a$ \\
    sheltering factor $P_m$     \\~\\
    Massman and Weil, \\
    \quad Boundary Layer Meteorology 91, 1999.
  \end{columns}
}

\frame{

  \frametitle{Plate Model}
  \begin{columns}[cc]
    \column{0.57\textwidth}
    \begin{center}
      \includegraphics[width=0.98\textwidth]{Images/patch4_plate_v2.eps}
    \end{center}
    \column{0.43\textwidth}
    \scriptsize
    Two-Stream Canopy Radiation \\
    Scattering \\~\\
    \[
    \begin{array}{l}
      \frac{dI_{du}}{dL} = \\
      \quad - f_1(\text{pft},\theta,\text{LAI}) I_{du} \\
      \quad + f_2(\text{pft},\theta,\text{LAI}) I_{dd}  \\
      \quad + f_3(\text{pft},\theta,\text{LAI}) e^{-KL} I_{b0} \\
    \end{array}
    \]
    Sellers, \\
    \quad Int. J. Remote Sensing, 6(8), 1985.
  \end{columns}
}

%\frame{
%  \begin{center}
%    \includegraphics[width=1.0\textwidth]{Images/2_canopies.eps}
%  \end{center}

%  \begin{columns}[cc]
%  \column{0.5\textwidth}
%  \scriptsize
%  Massman and Weil, \\
%  \quad Boundary Layer Meteorology 91, 1999.
%  \column{0.5\textwidth}
%  \scriptsize
%  Sellers, \\
%  \quad Int. J. Remote Sensing, 6(8), 1985.
%  \end{columns}
%}

%\subsection{Enthalpy Transfer}

%\frame{
%  \begin{center}
%    Design Objectives: \\~\\
%    \quad Conservative state variables \\
%    \quad Consistent energy scale across mediums \\
%    \quad Conserve phase change energy at different temperatures \\
%    \quad Conserve of liquid internal energy \\
%  \end{center}
%}

%\frame{
%  \begin{columns}[cc]
%    \column{0.4\textwidth}
%    \scriptsize
%    Vegetation State Equation:
%    \\
%    \[
%    \begin{array}{l}
%      H_{v}  =  \\
%      \quad f_{l} [ (m_{v}c_{i})T_3 + m_{v}L_{F(T_3)} + \\
%      \quad \quad (T_{v}-T_3)(m_{v}c_{l}) ] + \\
%      \quad (1-f_{l}) \left( (m_{v}c_{i})T_{v} \right) + \\
%      \quad T_{v} m_{b} c_{b} \\
%    \end{array}
%    \]
%    \\~\\
%    Canopy Air State Equation:
%    \[
%    \begin{array}{l}
%      H_c =  m_{dry} c_{pd} T_c + \\
%      \quad m_{vapor} \left(c_i T_3 + L_{S(T_3)} + c_{pv}(T_c-T_3)\right) \\
%    \end{array}
%    \]
    
%    \column{0.6\textwidth}
%    \begin{center}
%      \includegraphics[width=0.88\textwidth]{Images/enthalpy_diag.eps}
%    \end{center}
%  \end{columns}
%}

\frame{
\frametitle{Amorphous}
  \begin{center}
      \includegraphics[width=0.80\textwidth]{Images/canopy_diag_2.eps}
    \end{center}
}

\frame{
  \begin{columns}[cc]
    \column{0.4\textwidth}
    \scriptsize
    Vegetation Dynamic Equation:
    \[
    \begin{array}{rl}
      \frac{ \partial{H_{v(i)}}  }{ \partial{t}  }    = &   R_{S,net} \\
      & - R_{L,net} \\
      & - \dot{H}_{vc} \\
      & - \dot{H}_{TR(vapor)} \\
      & + \int \dot{H}_{TR,soil(z)} \\
      & + \dot{H}_{pint} \\
      & - \dot{H}_{ds} \\
    \end{array}
    \]
    \\~\\
    Canopy Dynamic Equation:
    \[
    \begin{array}{rl}
      \frac{\partial H_c}{\partial t} = & \dot{H}_{gc} + \dot{H}_{ac} \\
      & + \int \dot{H}_{vc} + \int \dot{H}_{TR} \\
      & + \dot{H}_{\Delta P} + \dot{H}_{\Delta \rho} \\
    \end{array}
    \]
    \column{0.6\textwidth}
    \begin{center}
      \includegraphics[width=0.88\textwidth]{Images/canopy_diag_4.eps}
    \end{center}
  \end{columns}
}

\subsection{Hyrbid Numerical Solvers - BDF2}

\frame{
\begin{columns}[c]
\column{0.50\textwidth}

\scriptsize
BDF2 - Trapezoidal Backwards Euler (implicit)
\[
  \frac{Y_{n+1}-Y_n}{\Delta t} + \frac{1}{2} \frac{Y_{n+1}-2Y_n+Y_{n-1}}{\Delta t} \approx \frac{\partial{Y_{n+1}}}{\partial{t}}
\]

\column{0.50\textwidth}
\includegraphics[height=0.6\textheight]{Images/bdf2_example.eps}
\end{columns}
}

\frame{
\scriptsize
\[
  \frac{Y_{n+1}-Y_n}{\Delta t} + \frac{1}{2} \frac{Y_{n+1}-2Y_n+Y_{n-1}}{\Delta t} \approx \frac{\partial{Y_{n+1}}}{\partial{t}}
\]
\\~\\
\[
\frac{\partial{Y_{n+1}}}{\partial{t}} = A \cdot Y_{n+1} + B
\]
\\~\\
\[
\begin{array}{lcl}
  Y_{n+1} = & \left[3I-2A\Delta t_+ \right]^{-1} \\
       & \left[ \left(3+\frac{\Delta t_+}{\Delta t_-}\right)Y_n 
    - \left(\frac{\Delta t_+}{\Delta t_-}\right)Y_{n-1} + 2B\Delta t_+\right] \\
\end{array}
\]
}

\frame{
\begin{columns}[cc]
\column{0.4\textwidth}
\scriptsize
\[
Y = \begin{bmatrix} H_c \\ H_{v(1)} \\ H_{v(2)} \\ ... \\ H_{v(n)} \end{bmatrix}
\]
\column{0.6\textwidth}
\includegraphics[width=0.85\textwidth]{Images/canopy_diag_4.eps}
\end{columns}
}


\frame{
\begin{columns}[cc]
\column{0.4\textwidth}
\scriptsize
\[
\frac{\partial{Y_{n+1}}}{\partial{t}} = A \cdot Y_{n+1} + B
\]
\\~\\
\[
\begin{array}{rl}
  \frac{ \partial{H_{v(i)}}  }{ \partial{t}  }    = &   R_{S,net} \\
  & - R_{L,net} \\
  & - \dot{H}_{vc} \\
  & - \dot{H}_{TR(vapor)} \\
  & + \int \dot{H}_{TR,soil(z)} \\
  & + \dot{H}_{pint} \\
  & - \dot{H}_{ds} \\
\end{array}
\]
\\~\\
\[
\begin{array}{rl}
  \frac{\partial H_c}{\partial t} = & \dot{H}_{gc} + \dot{H}_{ac} \\
  & + \int \dot{H}_{vc} + \int \dot{H}_{TR} \\
  & + \dot{H}_{\Delta P} + \dot{H}_{\Delta \rho} \\
\end{array}
\]
\column{0.6\textwidth}
\includegraphics[width=0.85\textwidth]{Images/canopy_diag_4.eps}
\end{columns}
}

\frame{
\begin{center}
\includegraphics[height=0.80\textheight]{Images/rkvshyb200_leafemp.eps}
\end{center}
}


\section{Future Directions}

\frame{

  \begin{columns}[c]
    \column{0.75\textwidth}
    \includegraphics[height=0.80\textheight]{Images/amap_012002_av.eps}
    \column{0.25\textwidth}
    \scriptsize
    Present Day (2008) \\
    Above Ground Biomass \\
    ``Spun-Up'' 500 years \\
  \end{columns}

}

\frame{
%  \frametitle{Model Structure - Challenges and Successes}
  \begin{itemize}
    \item{Best Axis for a Probabalistic Model?}
    \item{Characterization of Disturbance}
    \item{Global Scale PFT Parameterizations}
    \item{Speed, Stability, Accuracy}
    \item{Coordinating with other communities like CLM!}
  \end{itemize}
}

\frame{
\begin{center}
Thank you  \\
Discussion? Comments?
\end{center}

}





%\frame{
%\begin{itemize}
%  \item AV: mean annual precipitation deficit on the order of $15\%$
%  \item AV: decrease in total latent heat flux and total energy flux
%  \item AV: greater dynamic variability in upper column soil moisture
%  \item AV: lower leaf area promotes increased throughfall
%\end{itemize}
%}
%\subsection{Differential Hydrometeorology near Site 2, April 2003}

%\frame{
%  \begin{figure}[!ht]
%    \begin{center}
%      \includegraphics[height=0.95\textheight]{Images/diag_z3_apr03_pcpetmap.eps}
%    \end{center}
%  \end{figure}
%}

%\frame{
%\begin{columns}[c]
%\column{0.5\textwidth}
%\scriptsize
%\begin{table}[h!b!p!]
%  \begin{center}
%  \begin{tabular}{| c | c  c  c  c |}
%    \hline
%    Case    & $\Delta M_{pw}$ & $ET$ & $P$ & $Mc$ \\
%    \hline
%    AV      & -11.42 & 82.95  & 41.89 & -52.49 \\
%    PV      & -11.02 & 111.89 & 85.91 & -36.99 \\
%    \hline
%  \end{tabular}
%  \caption{Monthly mean sources and sinks of water mass for focus region 2.  Total change in column precipitable water $\Delta M_{pw}$, evapotranspiration $ET$, precipitation $P$ and resolved moisture convergence $Mc$ are spatial means integrated over the month of April 2003. Units are $[kg/m^2]$.}
%  \end{center}
%\end{table}

%\column{0.5\textwidth}

%\scriptsize
%\[\Delta M_{pw} = \text{ET} + Mc - P \]
%\[ Mc = F_{x-}-F_{x+} + F_{y-}-F_{y+} \]
%\includegraphics[width=0.99\textwidth]{Images/mc_flux_diag_crop.eps}
%\end{columns}
%}


\end{document}
